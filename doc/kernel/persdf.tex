\usepackage{fancyheadings}
\usepackage{amsmath}
\usepackage{amssymb}
\usepackage{psfig}
\usepackage{here}
\usepackage{array}
\usepackage{alltt}
\usepackage{graphicx}
\usepackage{eepic,epic}
% \usepackage[latin1]{inputenc}
% \usepackage[T1]{fontenc}
% \usepackage[french]{babel}
% \usepackage[dvips]{epsfig}

%\oddsidemargin -0.2cm
%\evensidemargin -0.2cm
%\topmargin -1cm
%\textheight 22.5cm
%\textwidth 16.2cm
%\headheight 1.0cm

\newfont{\eufmtwelve}   {eufm10 scaled \magstep1}
\newfont{\eufmten}      {eufm10 }
\newfont{\eufmnine}     {eufm9 }
\newfont{\eufmeight}    {eufm8 }
\newfont{\eufmseven}    {eufm7 }
\newfont{\eufmsix}      {eufm6 }
\newfont{\eufmfive}     {eufm5 }
\newfont{\eusmtwelve}   {eusm10 scaled \magstep1}
\newfont{\eusmten}      {eusm10}
\newfont{\eusmnine}     {eusm9 }
\newfont{\eusmeight}    {eusm8 }
\newfont{\eusmseven}    {eusm7 }
\newfont{\eusmsix}      {eusm6 }
\newfont{\eusmfive}     {eusm5 }
\newfont{\msbmtwelve}   {msbm10 scaled \magstep1}
\newfont{\msbmeight}    {msbm8}

\newcommand{\udl}{\underline}
\newcommand{\udll}[1]{{\udl{\udl{#1}}}}
\newcommand{\udlll}[1]{{\udl{\udl{\udl{#1}}}}}
\newcommand{\mat}[1]{{\mbox{\msbmtwelve {#1}}}}
\newcommand{\Reel}{{\mbox{\msbmtwelve R}}}      % L'ensemble des reels.
\newcommand{\reel}{{\mbox{\msbmeight R}}}       % L'ensemble des reels.
%\newcommand{\Reel}{{\rm I\hspace{-0.15em}R}}
\newcommand{\Complex}{\mbox{\msbmtwelve C}}     % L'ensemble des complexes.
\newcommand{\Naturel}{\mbox{\msbmtwelve N}}  % L'ensemble des entiers naturels.
\newcommand{\naturel}{\mbox{\msbmeight N}}   % L'ensemble des entiers naturels.

%\newcommand{\Naturel}{{\rm I\hspace{-0.15em}N}}% L'ensemble des entiers naturels.
\renewcommand{\emptyset}{\mbox{$\circ$\hspace{-.50em}/}}  % ensemble vide.
\newcommand{\Cont}{{\cal C}}            % L'ensemble des fonctions continues
\newcommand{\Cinf}{{\cal C}^{\infty}}   % L'ensemble des fonction C-infinies
\renewcommand{\vec}[1]{\overrightarrow{\!\!#1}}
\newcommand{\subsetcont}{{\subset\hspace{-.6em}_{\scriptscriptstyle >} }}
\newcommand{\Frac}[2]{{\ds \frac{\ds #1}{\ds #2}}}
\newcommand{\interior}[1]{{\stackrel{\circ}{#1}}}
\newcommand{\cqfd}{{$\mbox{}$\hfill\rule{2.5mm}{2.5mm}}}
\newcommand{\vectwo}[2]{{\left(\hspace{-.5em}\begin{array}{c} {#1} \\ {#2}
     \end{array}\hspace{-.5em}\right)}}
\newcommand{\vecthree}[3]{{\left(\hspace{-.5em}\begin{array}{c} {#1}
     \\ {#2} \\ {#3} \end{array}\hspace{-.5em}\right)}}
\newcommand{\vecfour}[4]{{\left(\hspace{-.5em}\begin{array}{c} {#1}
     \\ {#2} \\ {#3} \\ {#4} \end{array}\hspace{-.5em}\right)}}
\newcommand{\vecfive}[5]{{\left(\hspace{-.5em}\begin{array}{c} {#1}
     \\ {#2} \\ {#3} \\ {#4} \\ {#5} \end{array}\hspace{-.5em}\right)}}
\newcommand{\vecseven}[7]{{\left(\hspace{-.5em}\begin{array}{c} {#1}
     \\ {#2} \\ {#3} \\ {#4} \\ {#5} \\ {#6} \\ {#7} \end{array}\hspace{-.5em}\right)}}
\def\infess{\mathop{\iflanguage{english}{\mbox{ess$\,$inf}}{\mbox{inf$\,$ess}}}}
\def\supess{\mathop{\iflanguage{english}{\mbox{ess$\,$sup}}{\mbox{sup$\,$ess}}}}
\def\essinf{\mathop{\iflanguage{english}{\mbox{ess$\,$inf}}{\mbox{inf$\,$ess}}}}
\def\esssup{\mathop{\iflanguage{english}{\mbox{ess$\,$sup}}{\mbox{sup$\,$ess}}}}
\def\aplim{\mathop{\mbox{ap$\,$lim}}}
\def\aplimsup{\mathop{\mbox{ap$\,$lim$\,$sup}}}
\def\apliminf{\mathop{\mbox{ap$\,$lim$\,$inf}}}
\def\convto{\mathop{\hbox{\rightarrowfill}}} % converge vers.
\newcommand{\rightgap}{{]\hspace{-0.12em}]}}
\newcommand{\leftgap}{{[\hspace{-0.12em}[}}
\newcommand{\gapof}[1]{{\leftgap {#1} \rightgap}}
\newcommand{\restrictiona}[1]
{{ \begin{picture}(13,10) \put(-1,-4){$\mid_{#1}$} \end{picture}
}} % Le signe "Restriction sur #1"

\def\Indic{\mbox{1\hspace{-0.20em}I}}   % Fonction l'indicatrice

% \def\bar3{|\hspace{-1pt}\|} % 3bar verticaux pour les normes matricielles.
\def\cvweak{\mathop{-\hspace{-0.3em}-\hspace{-0.6em}\rightharpoonup}} % fleche cv faible
\def\cvweakstar{\cvweak^*} % fleche cv faible etoile
\def\longmapsto
{ \begin{picture}(0,10)
  \put(0,0){$\scriptstyle{\vdash}$} \end{picture} \mbox{$\longrightarrow$}
} 

\def\build#1_#2^#3{\mathrel{
 \mathop{\kern 0pt#1}\limits_{#2}^{#3}}} % Ecrire en dessous et dessus un symbole.

\def\Dist{\mbox{\eusmtwelve D}} %signe de distribution
\def\dist{\mbox{\eusmten D}} %signe de distribution


%definition de commandes utilises
\newcommand{\ds}{\displaystyle}
\newcommand{\rc}{{\par}}
\newcommand{\rcc}{{\par\medskip}}
\newcommand{\rccc}{{\par\bigskip}}


%definition des environnements theoreme, lemme, ...
\usepackage{boxedminipage}
% \newenvironment{largebox}
%   { \rc\noindent \begin{boxedminipage}[t]{\textwidth} }
%   { \end{boxedminipage}  \rccc\noindent }
\newenvironment{largebox}
  { \rc\noindent \begin{boxedminipage}[t]{\linewidth} }
  { \end{boxedminipage}  \rccc\noindent }


\newtheorem{ltheoreme}{Th\'eor\`eme}
\newenvironment{theoreme}
  { \begin{largebox} \begin{ltheoreme} }
  { \end{ltheoreme} \end{largebox} }
\newtheorem{lproposition}{Proposition}
\newenvironment{proposition}
  { \begin{largebox} \begin{lproposition} }
  { \end{lproposition} \end{largebox} }
\newtheorem{llemme}{Lemme}
\newenvironment{lemme}
  { \begin{largebox} \begin{llemme} }
  { \end{llemme} \end{largebox} }
\newtheorem{ldefinition}{D\'efinition}
\newenvironment{definition}
  { \begin{largebox} \begin{ldefinition} }
  { \end{ldefinition} \end{largebox} }
\newtheorem{lhypothese}{Hypoth\`ese}
\newenvironment{hypothese}
  { \begin{largebox} \begin{lhypothese} }
  { \end{lhypothese} \end{largebox} }
\newtheorem{lcorollaire}{Corollaire}
\newenvironment{corollaire}
  { \begin{largebox} \begin{lcorollaire} }
  { \end{lcorollaire} \end{largebox} }
\newenvironment{remarque}
  { \begin{largebox} {\bf \udl{Remarque} : }}
  { \end{largebox} }

\newcounter{numberofprobl}
\setcounter{numberofprobl}{1}

\newlength{\compteurtpourprobla}
\newlength{\compteurtpourproblb}
\newenvironment{caseeqnarray}[1]
  {
   $${#1}
   \settowidth{\compteurtpourprobla}{${#1}\left\{\right.$}
   \setlength{\compteurtpourproblb}{\textwidth}
   \addtolength{\compteurtpourproblb}{-1\compteurtpourprobla}
   \settowidth{\compteurtpourprobla}{$\;$}
   \addtolength{\compteurtpourproblb}{-1\compteurtpourprobla}
   \left\{ \begin{minipage}[l]{\compteurtpourproblb}
   \vspace{-1em} \begin{eqnarray}
  }
  { \end{eqnarray} \end{minipage} \right. $$}


\newtheorem{hypothesis}{Hypothesis}
\newtheorem{prop}{Proposition}
\newtheorem{defi}{Definition}
%\newtheorem{theorem}{Theorem}
%\newtheorem{lemma}{Lemma}


% pour plus tard ...
% \DeclareGraphicsRule{ps.Z}{eps}{ps.bb}{`zcat #1}
% \DeclareGraphicsRule{eps.Z}{eps}{eps.bb}{`zcat #1}
% \DeclareGraphicsRule{ps.gz}{eps}{ps.bb}{`gunzip #1}
% \DeclareGraphicsRule{eps.gz}{eps}{eps.bb}{`gunzip #1}
